\documentclass[oneside,11pt,dvipsnames]{book}

\usepackage[utf8]{inputenc}
\usepackage[T1]{fontenc}
\usepackage[margin=1.5in]{geometry}
\usepackage{soul}
% \usepackage[small,compact]{titlesec} %very powerful
\usepackage[most]{tcolorbox}
% \setsecnumdepth{subsection}
% \setcounter{tocdepth}{3}
\usepackage{enumitem}
\usepackage{epigraph}
\usepackage{cite}
\usepackage{caption}
\captionsetup{font=small}
\usepackage{graphicx}
\usepackage{pdfpages}
\usepackage{hyperref}
\usepackage{wrapfig}
\setlength\intextsep{0pt} % remove extra space above and below in-line float
\usepackage{hyperref}
\hypersetup{
  colorlinks,
  citecolor=black,
  filecolor=black,
  linkcolor=blue,
  urlcolor=blue,
}
\usepackage{booktabs}


\usepackage{tikz}
\usetikzlibrary{calc}
\usepackage{xcolor}

\usepackage{anyfontsize}
\usepackage{sectsty}

\usepackage[makeroom]{cancel}

\newtcolorbox{mybox}{
  enhanced,
  boxrule=0pt,frame hidden,
  borderline west={2pt}{0pt}{green!75!black},
  colback=green!10!white,
  sharp corners
}

\newenvironment{commentbox}[1][]{
  \small
  \begin{mybox}
    {\small \textbf{#1}}
  }{
  \end{mybox}
}

\newtcolorbox{mydomesticbox}{
  enhanced,
  boxrule=0pt,frame hidden,
  borderline west={2pt}{0pt}{red!75!black},
  colback=blue!10!white,
  sharp corners
}

\newenvironment{domesticbox}[1][]{
  \small
  \begin{mydomesticbox}
    {\small \textbf{#1}}
  }{
  \end{mydomesticbox}
}

\renewcommand{\figurename}{Fig.}
\renewcommand{\tablename}{Tab.}
\def\Section{\S}
\renewcommand{\figureautorefname}{Fig.}
\renewcommand{\tableautorefname}{Tab.}
\makeatletter
\renewcommand{\chapterautorefname}{\S\@gobble}
\renewcommand{\sectionautorefname}{\S\@gobble}
\renewcommand{\subsectionautorefname}{\S\@gobble}
\renewcommand{\appendixautorefname}{\S\@gobble}
\makeatother

\newcommand{\mycomment}[3][\color{blue}]{{#1{{#2}: {#3}}}}
\newcommand{\tvn}[1]{\mycomment{TVN}{#1}}{}
\newcommand{\didi}[1]{\mycomment{Didier}{#1}}{}
\newcommand{\tl}[1]{\mycomment{ThanhLe}{#1}}{}
\newcommand{\red}[1]{{\color{red}{#1}}}
\newcommand{\xz}[1]{\mycomment{Xiaokuan}{[#1]}}{}


\begin{document}

\pagestyle{empty}
\begin{tikzpicture}[overlay,remember picture]

    % Background color
    \fill[
    black!2]
    (current page.south west) rectangle (current page.north east);
    
    % Rectangles
    \shade[
    left color=Dandelion, 
    right color=Dandelion!40,
    transform canvas ={rotate around ={45:($(current page.north west)+(0,-6)$)}}] 
    ($(current page.north west)+(0,-6)$) rectangle ++(9,1.5);
    
    \shade[
    left color=lightgray,
    right color=lightgray!50,
    rounded corners=0.75cm,
    transform canvas ={rotate around ={45:($(current page.north west)+(.5,-10)$)}}]
    ($(current page.north west)+(0.5,-10)$) rectangle ++(15,1.5);
    
    \shade[
    left color=lightgray,
    rounded corners=0.3cm,
    transform canvas ={rotate around ={45:($(current page.north west)+(.5,-10)$)}}] ($(current page.north west)+(1.5,-9.55)$) rectangle ++(7,.6);
    
    \shade[
    left color=orange!80,
    right color=orange!60,
    rounded corners=0.4cm,
    transform canvas ={rotate around ={45:($(current page.north)+(-1.5,-3)$)}}]
    ($(current page.north)+(-1.5,-3)$) rectangle ++(9,0.8);
    
    \shade[
    left color=red!80,
    right color=red!80,
    rounded corners=0.9cm,
    transform canvas ={rotate around ={45:($(current page.north)+(-3,-8)$)}}] ($(current page.north)+(-3,-8)$) rectangle ++(15,1.8);
    
    \shade[
    left color=orange,
    right color=Dandelion,
    rounded corners=0.9cm,
    transform canvas ={rotate around ={45:($(current page.north west)+(4,-15.5)$)}}]
    ($(current page.north west)+(4,-15.5)$) rectangle ++(30,1.8);
    
    \shade[
    left color=RoyalBlue,
    right color=Emerald,
    rounded corners=0.75cm,
    transform canvas ={rotate around ={45:($(current page.north west)+(13,-10)$)}}]
    ($(current page.north west)+(13,-10)$) rectangle ++(15,1.5);
    
    \shade[
    left color=ForestGreen,
    rounded corners=0.3cm,
    transform canvas ={rotate around ={45:($(current page.north west)+(18,-8)$)}}]
    ($(current page.north west)+(18,-8)$) rectangle ++(15,0.6);
    
    \shade[
    left color=ForestGreen,
    rounded corners=0.4cm,
    transform canvas ={rotate around ={45:($(current page.north west)+(19,-5.65)$)}}]
    ($(current page.north west)+(19,-5.65)$) rectangle ++(15,0.8);
    
    \shade[
    left color=OrangeRed,
    right color=red!80,
    rounded corners=0.6cm,
    transform canvas ={rotate around ={45:($(current page.north west)+(20,-9)$)}}] 
    ($(current page.north west)+(20,-9)$) rectangle ++(14,1.2);
    
 
    
    % Title
    \node[align=center] at ($(current page.center)+(0,-5)$) 
    {
    {\fontsize{38}{1} \selectfont {{Neural Network Verification}}}\\[0.5in]
    
    {\fontsize{14}{19.2} \selectfont \textcolor{ForestGreen}{ \bf ThanhVu (Vu) Nguyen}}\\[0.1in]
    \today{} (latest version available on  \href{https://github.com/nguyenthanhvuh/phd-cs-us}{Github})
    };
    \end{tikzpicture}

    
\chapter*{Preface}
Having been involved in PhD admission committees for many years, I've realized that many \textbf{international} students, especially those in smaller countries or less well-known universities, lack a clear understanding of
the Computer Science PhD admission process at US universities. This confusion not only
discourages students from applying but also creates the perception that
getting admitted to a CS PhD program in the US is difficult compared to other countries.

% though \emph{very} top schools could be very selective, e.g., see the \href{https://da-data.blogspot.com/2015/03/reflecting-on-cs-graduate-admissions.html}{admission process} at CMU
So I want to share some details about the admission process and advice for those who are interested in applying for a \textbf{PhD in Computer Science in the US}.
Originally, this document was intended for international students, but I have expanded it to include information that might also be useful for \emph{US domestic students}.
Moreover, while this is primarily intended for students interested in CS, it might be relevant to students from various STEM (Science, Technologies, Engineering, and Mathematics) disciplines.
Furthermore, although many examples are specifics for schools that I and other contributors of this document know about, the information should be generalizable to other R1\footnote{An \href{https://en.wikipedia.org/wiki/List_of_research_universities_in_the_United_States}{R1 institution} in the US is a research-intensive university with a high level of research activity across various disciplines. Currently, 146 (out of 4000) US universities are classified as R1.} institutions in the US.

This information can also help \textbf{US faculty and admission committee} gain a better understanding of international students and their cultural differences.  By recognizing and leveraging these differences, CS programs in the US can attract larger and more competitive application pools from international students.

I wish you the best of luck. Happy school hunting!

\begin{mybox}
This document will be updated regularly to reflect the latest information and updates in the admission process. Its latest version is available at

\begin{center}
  \href{https://nguyenthanhvuh.github.io/phd-cs-us/demystify.pdf}{nguyenthanhvuh.github.io/phd-cs-us/demystify.pdf},
\end{center}

\noindent and its \LaTeX{} source is also on \href{https://github.com/nguyenthanhvuh/phd-cs-us}{GitHub}. If you have questions or comments, feel free to create new \href{https://github.com/nguyenthanhvuh/phd-cs-us/issues}{GitHub issues} or \href{https://github.com/nguyenthanhvuh/phd-cs-us/discussions}{discussions}.

\end{mybox}

\newpage
\tableofcontents

\chapter{Summary}\label{sec:summary}

\bibliographystyle{abbrv}
\bibliography{demystify.bib}

\end{document}
